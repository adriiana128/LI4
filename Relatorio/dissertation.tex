
% example for dissertation.sty
\documentclass[
  % Replace oneside by twoside if you are printing your thesis on both sides
  % of the paper, leave as is for single sided prints or for viewing on screen.
  oneside,
  %twoside,
  11pt, a4paper,
  footinclude=true,
  headinclude=true,
  cleardoublepage=empty
]{scrbook}

\usepackage{dissertation}



% ACRONYMS -----------------------------------------------------

%import the necessary package with some options
\usepackage[acronym,nonumberlist,nomain]{glossaries}

%enable the following to avoid links from the acronym usage to the list
%\glsdisablehyper

%displays the first use of an acronym in italic
\defglsdisplayfirst[\acronymtype]{\emph{#1#4}}

%the style of the Glossary
\glossarystyle{listgroup}

% set the name for the acronym entries page
\renewcommand{\glossaryname}{Acronyms}

%this shall be the last thing in the acronym configuration!!
\makeglossaries


% here are the acronym entries
\newacronym{mei}{MEI}{Mestrado em Engenharia Informática}

    
% these could go in an acronyms.tex file, and loaded with:
% \loadglsentries[\acronymtype]{Parts/Definitions/acronyms}
% when using this, you may want to remove 'nomain' from the package options

%% **MORE INFO** %%

%to add the acronyms list add the following where you want to print it:
%\printglossary[type=\acronymtype]
%\clearpage
%\thispagestyle{empty}

%to use an acronym:
%\gls{qps}

% compile the thesis in command line with the following command sequence:
% pdlatex dissertation.tex
% makeglossaries dissertation
% bibtex dissertation
% pdlatex dissertation.tex
% pdlatex dissertation.tex

% ----------------------------------------------------------------

% Title
\titleA{Title1}
\titleB{Title2} % (if any)

% Author
\author{Aluno 1}

% Supervisor(s)
\supervisor{Orientador}
\cosupervisor{Coorientador}

% Date
\date{\myear} % change to text if date is not today

% Keywords
%\keywords{master thesis}

% Glossaries & Acronyms
%\makeglossaries  %  either use this ...
%\makeindex	   % ... or this

% Define Acronyms
%%!TEX root = ../dissertation.tex

\newacronym{mei}{MEI}{Mestrado em Engenharia Inform\'{a}tica}
%\glsaddall[types={\acronymtype}]

\ummetadata % add metadata to the document (author, publisher, ...)

\begin{document}
	% Cover page ---------------------------------------
	\umfrontcover	
	\umtitlepage
	
	% Add acknowledgements ----------------------------
	\chapter*{Acknowledgements}
	Write acknowledgements here


	% Add abstracts (en,pt) ---------------------------

	\chapter*{Abstract}
	\input{abstract/abstractEN.tex}
	
	\cleardoublepage
	\chapter*{Resumo}
	\input{abstract/abstractPT.tex}
	
	
	% Summary Lists ------------------------------------
	\tableofcontents
	\listoffigures
	\listoftables
	%\lstlistoflistings
	%\listofabbreviations
	\printglossary[type=\acronymtype]
	\clearpage
	\thispagestyle{empty}

	
	\pagenumbering{arabic}
	
% CHAPTER - Introdução -------------------------------------------------------------------------------
	\chapter{Introduction}
	    Introdução
        \section{Context}
        \paragraph{
A  contexto em que se insere esta apicação está associado a uma acção cada vez mais frequente na atualidade. Toda a gente gosta de registar os momentos e partilhar estes mesmos com amigos familiares e a comunidade.
}
\paragraph{
Temos como o exemplo das inúmeras redes sociais que têm sido desenvolvidas nos últimos anos. Assim como inúmeras empresas que recorrem à fotografia para demonstrar/representar uma actividade, local, etc.
}
\paragraph{
Mesmo as entidades que fabricam vários dispositivos como é o exemplo dos Smartphones, estão cada vez mais direcionadas para melhorar a qualidade das suas camêras fotográficas.
}
\paragraph{
Complementando com o que foi referido anteriormente, hoje em dia cada vez mais pessoas gostam de visitar não só localmente como internacionalmente vários Países, cidades ou até mesmo monumentos.
}
\paragraph{
A nossa aplicação baseia-se na possibilidade da partilha de uma fotografia associada a um ponto geográfico(local), que pode referenciar não só uma cidade como um local mais específico que pertence a essa mesma cidade. Permitindo também aos utilizadores deixarem uma classificação a essas mesmas fotografias de forma a ter uma lista/tabela classificativa por um determinado parâmetro, como por exemplo por cidade.
}
        \section{Motivation}
        \paragraph{
O propósito desta aplicação visa a partilha de fotografias num determinado local, com um sistema de classificação atribuída pelos seus utilizadores. Sendo que os utilizadores não só podem consultar a qualidade fotográfica de um utilizador, como o ficar a conhecer, melhor, o próprio local que pretendem visitar.
}
\paragraph{
Após levantamento de alguns dados estatístico relativamente ao nosso público alvo apresenta-se que:
}
\paragraph{
Redes sociais:
\newline
\textbullet 2.5+ biliões de utilizadores ativos no Facebok.
\newline
\textbullet 1+ biliões de utilizadores ativos no Instagram.
}
\paragraph{
Número de viagens:
\newline
\textbullet 1.4 biliões de chegadas internacionais, sendo que por vezes uma pessoa efetua mais do que 1 viagem decidiu-se reduzir este número para metade.
\newline
\textbullet A nivel doméstico estima-se que seja num número maior pois são efectuadas com mais frequência.
}
\paragraph{
Sites destinados a viagens:
\newline
\textbullet 17+ milhões de visualizações efetuadas por mês.
}
\paragraph{
\textbullet Aplicações direcionadas a fotografia:
\newline
\textbullet 1+ bilião de votos por mês www.gurushots.com.
\newline
\textbullet 120 mil utilizadores ativos e com um número de crescimento de 6 mil por mês www.photocrowd.
}
\paragraph{
Após a análise estatistica efetuado verificamos que existe um inúmero público que poderá ser um futuro utilizador da nossa aplicação. E que existem várias entidades com as quais poderemo-nos associar de forma a divulgar a nossa aplicação.
}
        \section{Goals}
        Objetivos
        \section{Structure}
        \input{introduction/structure.tex}
		

% CHAPTER - Estado da Arte ---------------------------------------------------------------------------
	\chapter{Literature Review}
		State of the art review; related work
        Introdção
        Section 1
        \section{Section 1}

        
        \section{Basics/Background/Related work}



	% CHAPTER - Problem and Challenges ---------------
	\chapter{Methods}
	         The problem and its challenges.

	\section{Proposed Approach - solution}
	In this section, it is presented various ways to display an image.
     \subsection{System Architecture}
     A block diagram of the planned system / approach

	Here we have an example of inserting an image between the text paragraphs.
	
	% CHAPTER - Contribution -------------------------
	\chapter{Development}
		
	\section{Decisions}
    \section{Implementation}
    \section{Outcomes}
    Main result(s) and their scientific evidence
	\section{Summary}


	% CHAPTER - Application -------------------------
	\chapter{Case Studies / Experiments}
		Application of main result (examples and case studies)
	\section{Experiment setup}
    \section{Results}
    \section{Discussion}
	\section{Summary}

	% CHAPTER - Conclusion/Future Work --------------
	\chapter{Conclusion}
		Conclusions and future work.
	\section{Conclusions}
	\section{Prospect for future work}
			
	\bookmarksetup{startatroot} % Ends last part.
	\addtocontents{toc}{\bigskip} % Making the table of contents look good.
	%\cleardoublepage

	%- Bibliography (needs bibtex) -%
	\bibliography{dissertation}

	% Index of terms (needs  makeindex) -------------
	%\printindex
	
	
	% APPENDIX --------------------------------------
	\umappendix{Appendix}
	
	% Add appendix chapters
	\chapter{Support material}
	Auxiliary results which are not main-stream; or

	%\chapter{Details of results}
	Details of results whose length would compromise readability of main text; or

	%\chapter{Listings}
	Specifications and Code Listings: should this be the case; or

	%\chapter{Tooling}
	Tooling: Should this be the case.

	%Anyone using \Latex\ should consider having a look at \TUG,
	%the \tug{\TeX\ Users Group}


	% Back Cover -------------------------------------------
	\umbackcover{
	NB: place here information about funding, FCT project, etc in which the work is framed. Leave empty otherwise.
	}


\end{document}
